\chapter{State of Research}
\label{chap:research}

\section{Existing Approaches to Validation and Verification of Digital Twins}
% Manual vs. automated approaches
% Critical discussion of existing V&V definitions and methods (→ Reference back to 2.2)
% Challenges in the validation of automatically generated models

\section{Automatic Model Generation for Digital Twins}
% Advantages of automatic generation:
% - Continuous up-to-dateness and online validation (→ Reference to 4.5)
% - Coping with high levels of complexity
% - Standardization and transparency
% - Scalability
% - Avoidance of bias in manual validations
% - Cost savings
% Challenges in automatic model generation (→ Reference to problem definition in 1.2)

\section{Machine Learning-Based Approaches for Model Validation}
% Classification methods for the detection of model deviations (→ Reference to 4.3)
% Challenges in data preparation and feature selection
% Discussion of previous ML approaches in the context of the V&V problem (→ Reference to 2.2 and 4.3)

\section{Research Gaps and Open Questions}
% Proposed research approach (Abduction and Design Science Research)
% Identification of the research gap on automated V&V (→ Reference to research questions in 1.3)
% Delimitation of own research approach (→ Transition to Chapter 4)
