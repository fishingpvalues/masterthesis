% Description: Contains the abstract of the thesis.
\thispagestyle{plain} % remove headers for the abstract page if desired

\vspace*{1.5in} % added vertical space to lower the abstract on the page

\begin{center}
  {\Large \textbf{Abstract}}
\end{center}

This thesis addresses the validation of automatically generated Simulation-Based Digital Twins (SBDTs) within discrete material flow systems (DMFS). While learning SBDTs from data promises reduced creation and updating efforts, these benefits are negated if Verification, Validation, and Uncertainty Quantification (VVUQ) require manual expert involvement. To overcome this, a multi-layered, automated VVUQ framework is developed and empirically validated. This framework integrates data processing using Object-Centric Event Logs, twin synchronization, and Machine Learning-based validation using a supervised classification approach. Results from a conducted Internet of Things (IoT) Factory case study demonstrate that a Bidirectional Long Short-Term Memory (BiLSTM)-based classifier effectively identifies statistically significant differences between real and simulated process data, yielding consistently high rejection rates and highly significant p-values ($p \ll 0.01$) in permutation tests across multiple SBDT components (process flow, resource allocation, time models). Permutation testing confirms these findings across multiple SBDT components (process flow, resource allocation, time models), revealing specific areas where the current SBDT configuration lacked fidelity in the case study. The framework enables continuous, automated validation, offering a scalable and objective alternative to periodic manual checks and providing feedback for targeted SBDT improvement. While requiring initial infrastructure setup efforts, the methodology contributes a practical approach for enhancing SBDT reliability and trustworthiness in manufacturing.

\vspace{0.3in}

\textbf{Keywords:} Simulation-Based Digital Twins (SBDT), Automated Verification, Validation, and Uncertainty Quantification (VVUQ), Data-Driven Validation, Object-Centric Event Log (OCEL), Machine Learning, Permutation Testing, Discrete Material Flow Systems (DMFS), Manufacturing Systems, Process Mining.

\clearpage