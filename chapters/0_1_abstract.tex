% Description: Contains the abstract of the thesis.
\thispagestyle{plain} % remove headers for the abstract page if desired

\vspace*{1.5in} % added vertical space to lower the abstract on the page

\begin{center}
  {\Large \textbf{Abstract}}
\end{center}

This thesis investigates the application of Digital Twins within discrete material flow systems, a key component of Industry 4.0. It reviews the progression from Digital Models and Digital Shadows to fully realized Digital Twins that integrate real-time data, simulation, and control mechanisms. To learn these Digital Twins from data enables companies to reduce twin creation and updating efforts. These gained advantages would be made obsolete if the validation and verification of such twins were performed manually involving experts. A multi-layered framework integrating data processing, twin synchronization, VVUQ processes, and decision support is developed and empirically validated. Results from the IoT Factory case study confirm that the BiLSTM-based classification approach successfully identifies statistically significant discrepancies between real and simulated processes with higher sensitivity than traditional validation methods. Permutation testing validates these findings across multiple SBDT components, including process flow, resource allocation, and time models. The framework enables continuous, automated validation while providing actionable insights to stakeholders, overcoming limitations of periodic manual validation. This research contributes to both theoretical understanding of VVUQ requirements for automatically generated models and practical implementation of data-driven validation techniques for manufacturing systems.

\medskip

\textbf{Keywords:} Digital Twins, Automated VVUQ, Process Mining, Machine Learning, Discrete Material Flow Systems.

\clearpage