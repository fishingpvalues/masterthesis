% Description: Contains the abstract of the thesis.
\thispagestyle{plain}

\vspace*{1.5in}

\begin{center}
  {\Large \textbf{Abstract}}
\end{center}

This thesis addresses the validation of automatically generated simulation-based digital twins within discrete material flow systems. While generating simulation-based digital twins from data promises reduced creation and updating efforts, these benefits are negated if verification, validation, and uncertainty quantification (VVUQ) require manual expert involvement. To overcome this, a multi-layered, automated VVUQ framework is developed and empirically validated. This framework integrates data processing using object-centric event logs, twin synchronization, and machine learning-based validation via a supervised classification approach. Results from a conducted case study demonstrate that a Bidirectional Long Short-Term Memory-based classifier effectively identifies statistically significant differences between real and simulated process data, yielding consistently high rejection rates and highly significant p-values ($p \ll 0.01$) in permutation tests across multiple simulation-based digital twin components (process flow, resource allocation, time models). The framework enables continuous, automated validation, offering a scalable and objective alternative to periodic manual checks and providing feedback for targeted simulation-based digital twin improvement. While requiring initial infrastructure setup efforts, the methodology contributes a practical approach for enhancing simulation-based digital twin reliability and trustworthiness in manufacturing.

\vspace{0.3in}

\textbf{Keywords:} Simulation-based digital twin, Automated verification, validation, and uncertainty quantification, Data-driven validation, Object-centric event log, Machine learning, Permutation testing, Discrete material flow systems, Manufacturing systems, Process mining.

\clearpage