\chapter{Framework Design}
\label{chap:methodology}

\section{Requirements Analysis}
% Functional requirements
% Technical requirements
% Requirements for data formats
% Derivation of requirements from theoretical findings (→ reference to 2.2 and 3.4)

\section{Data-Based Validation Strategy}
% Object-centric event logs as a basis for validation (→ Reference to 2.3)
% Temporal data partitioning (training, validation, test)
% Feature selection: identification of relevant features from event logs
% Order reconstruction from process data
% Preprocessing and Data Cleaning: Handling missing data, unifying part IDs, and filtering out irrelevant process steps (→ Reference to 6.1)
% Methodical implementation of the theoretical V&V concepts (→ Reference to 2.2)

\section{Machine Learning-Based Validation Approach}
\label{sec:ml_validation}
% Simulation model for the generation of synthetic event logs
% Classification-based deviation detection as a unified V&V approach (→ Reference to 2.2 and 3.3)
% Feature Engineering: Creation of features such as "duration," "sequence_number," and periodic time features (e.g., "day_of_week_sin," "hour_of_day_cos") (→ Reference to 6.3)
% Hyperparameter optimization and model selection
% Artificial error injection to validate the approach (→ Reference to 6.3)
% Clustering and Anomaly Detection: Use of KMeans clustering to identify patterns and anomalies in the data (→ Reference to 6.4)

\section{Metrics and Key Figures for Model Evaluation}
\label{sec:metrics}
% Process-oriented metrics (throughput times, resource utilization) (→ Reference to 2.4)
% Time-related metrics (start and end times, processing times)
% Order-related metrics (completeness, sequence fidelity)
% Derivation of metrics from V&V theory (→ Reference to 2.2)
% F1 Score, ROC AUC, and other classification metrics for model evaluation (→ Reference to 6.3)

\section{Online Validation and Continuous Monitoring}
% Early detection of model deviations
% Recommendations for action in the event of model drift
% V&V as a continuous process