\chapter{Conclusion and Future Work} % Renamed slightly
\label{chap:conclusion}

\section{Summary of Key Findings and Contributions}
\label{sec:conclusion_summary_findings} % Combines your Summary and parts of Insights
% Directly and concisely answer the research questions (RQ1, RQ2, RQ3) based on the discussion in Chapter \ref{chap:discussion}.
% State whether your initial hypotheses (if formally stated) were supported.
% Briefly synthesize the main empirical results (e.g., framework implementation, successful detection of discrepancies by DT/BiLSTM, comparison outcomes).
% Clearly state the main contributions of the thesis (e.g., the developed framework itself, the methodological insights for automated VVUQ).

\section{Concluding Remarks}
\label{sec:conclusion_remarks}
% Provide a final, high-level perspective on the research.
% - Reiterate the significance of addressing the VVUQ challenge for SBDTs.
% - Briefly reflect on the overall success and impact of the proposed approach.
% - Offer a final concluding thought on the value of data-driven validation.

\section{Future Research Directions}
\label{sec:conclusion_future_work} % Renamed from "Outlook"
% Based on the discussion (\autoref{sec:discussion_limitations}), outline specific avenues for future work.
% Your placeholder points fit well here:
% - Potential for further developments (e.g., enhancing the framework with XAI, UQ, online capabilities).
% - Open questions and further research needs (e.g., application to different domains/scales, integration standards).
% - Specific suggestions for follow-up studies.