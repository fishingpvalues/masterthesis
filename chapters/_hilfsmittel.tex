\chapter*{Übersicht verwendeter Hilfsmittel}
\thispagestyle{plain}

\section*{Generative KI-Werkzeuge}

\begin{tabular}{p{3cm}p{5cm}p{7cm}}
  \textbf{Tool}   & \textbf{Zugriffsquelle}                   & \textbf{Verwendungszweck}                                                                                                                                                                                                                  \\
  \hline
  GitHub Copilot  & \url{https://github.com/features/copilot} &
  Formatierung und Einrückung von Code. Nutzung als Debugging Unterstützung für die schnellere Erkennung von Fehlern im Code. Generierung von \textbf{Dokumentationsunterlagen} wie README, Code Kommentaren und Lizenzen. Anwendung von Vorschlägen für die Architektur des Frameworks                    \\
  \hline
  ChatGPT         & \url{https://chat.openai.com}             &
  Formulierungshilfe für englische Textpassagen, Korrekturvorschläge für Grammatik, Umformulierung komplexer Sätze. Rechtschreibunterstützung. Anfertigung von mathematischen Zeichungen und PlantUML Diagrammen auf Basis von Bildern. Erstellung von Formeln in \LaTeX auf Basis von Paperinformationen. \\
  \hline
  DeepL           & \url{https://deepl.com}                   &
  Überprüfung der englischen Grammatik und Rechtschreibung im gesamten Dokument.                                                                                                                                                                                                                           \\
  \hline
  Research Rabbit & \url{https://researchrabbitapp.com/}      &
  Literaturrecherche und Management.                                                                                                                                                                                                                                                                       \\
  \hline
\end{tabular}

\section*{Verwendungsumfang}

Die generativen KI-Werkzeuge wurden wie folgt eingesetzt:

\begin{itemize}
  \item \textbf{GitHub Copilot}: Hauptsächlich zur Unterstützung bei Implementierungsdetails im Python-Code für die Datenverarbeitung und Modellierung. Alle generierten Vorschläge wurden manuell überprüft, verstanden und bei Bedarf angepasst. Verwendung in ca. 35\% der gesamten Codeimplementierung.

  \item \textbf{ChatGPT}: Zur sprachlichen Verbesserung komplexer wissenschaftlicher Formulierungen, insbesondere bei methodologischen Beschreibungen und Diskussionen. Alle Formulierungsvorschläge wurden kritisch geprüft und in den Kontext der eigenen Argumentation integriert. Unterstützung bei ca. 30\% des geschriebenen Texts.

  \item \textbf{DeepL}: Überprüfung der finalen Version auf grammatikalische und stilistische Fehler. Vorgeschlagene Korrekturen wurden manuell überprüft und nur bei Erhaltung des beabsichtigten wissenschaftlichen Inhalts übernommen.
  \item \textbf{Research Rabbit}: Unterstützung bei der Literaturrecherche und -verwaltung.
\end{itemize}

\section*{Kennzeichnung generierter Inhalte}

Alle durch KI-Werkzeuge vorgeschlagenen Formulierungen, die substantiell in den Text übernommen wurden, sind durch das Symbol \ai{Text} gekennzeichnet. Geringfügige grammatikalische oder stilistische Anpassungen wurden nicht separat markiert, da sie den Inhalt nicht wesentlich verändert haben.

Inhalte aus wissenschaftlichen Quellen und eigene Gedanken stellen den überwiegenden Teil der vorliegenden Arbeit dar. Die Verwendung generativer Werkzeuge diente ausschließlich der sprachlichen und technischen Unterstützung und hat die eigene kritische Auseinandersetzung mit dem Forschungsthema nicht ersetzt.
