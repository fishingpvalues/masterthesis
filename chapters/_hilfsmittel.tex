\chapter*{Übersicht verwendeter Hilfsmittel}
\thispagestyle{plain}

\section*{Generative KI-Werkzeuge}

\begin{tabular}{p{3cm}p{5cm}p{7cm}}
  \textbf{Tool}  & \textbf{Zugriffsquelle}                   & \textbf{Verwendungszweck}                                                                   \\
  \hline
  GitHub Copilot & \url{https://github.com/features/copilot} &
  Unterstützung bei der Programmierung von Python-Code für Datenverarbeitung und Modellimplementierung. Vorschläge für Codeergänzungen und Lösungsansätze. \\
  \hline
  ChatGPT        & \url{https://chat.openai.com}             &
  Formulierungshilfe für englische Textpassagen, Korrekturvorschläge für Grammatik, Umformulierung komplexer Sätze.                                        \\
  \hline
  Grammarly      & \url{https://app.grammarly.com}           &
  Überprüfung der englischen Grammatik und Rechtschreibung im gesamten Dokument.                                                                           \\
  \hline
\end{tabular}

\section*{Verwendungsumfang}

Die generativen KI-Werkzeuge wurden wie folgt eingesetzt:

\begin{itemize}
  \item \textbf{GitHub Copilot}: Hauptsächlich zur Unterstützung bei Implementierungsdetails im Python-Code für die Datenverarbeitung und Modellierung. Alle generierten Vorschläge wurden manuell überprüft, verstanden und bei Bedarf angepasst. Verwendung in ca. 35\% der gesamten Codeimplementierung.

  \item \textbf{ChatGPT}: Zur sprachlichen Verbesserung komplexer wissenschaftlicher Formulierungen, insbesondere bei methodologischen Beschreibungen und Diskussionen. Alle Formulierungsvorschläge wurden kritisch geprüft und in den Kontext der eigenen Argumentation integriert. Unterstützung bei ca. 30\% des geschriebenen Texts.

  \item \textbf{Grammarly}: Überprüfung der finalen Version auf grammatikalische und stilistische Fehler. Vorgeschlagene Korrekturen wurden manuell überprüft und nur bei Erhaltung des beabsichtigten wissenschaftlichen Inhalts übernommen.
\end{itemize}

\section*{Kennzeichnung generierter Inhalte}

Alle durch KI-Werkzeuge vorgeschlagenen Formulierungen, die substantiell in den Text übernommen wurden, sind durch Fußnoten gekennzeichnet. Geringfügige grammatikalische oder stilistische Anpassungen wurden nicht separat markiert, da sie den Inhalt nicht wesentlich verändert haben.

Inhalte aus wissenschaftlichen Quellen und eigene Gedanken stellen den überwiegenden Teil der vorliegenden Arbeit dar. Die Verwendung generativer Werkzeuge diente ausschließlich der sprachlichen und technischen Unterstützung und hat die eigene kritische Auseinandersetzung mit dem Forschungsthema nicht ersetzt.
