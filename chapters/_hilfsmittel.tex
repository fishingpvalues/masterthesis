\chapter*{Übersicht verwendeter Hilfsmittel}
\thispagestyle{plain}

\section*{Generative KI-Werkzeuge}

\begin{tabular}{p{3cm}p{5cm}p{7cm}}
  \textbf{Tool}   & \textbf{Zugriffsquelle}                   & \textbf{Verwendungszweck}                                                                                                                                                                                                                  \\
  \hline
  GitHub Copilot  & \url{https://github.com/features/copilot} &
  Formatierung und Einrückung von Code. Nutzung als Debugging Unterstützung für die schnellere Erkennung von Fehlern im Code. Generierung von \textbf{Dokumentationsunterlagen} wie README, Code Kommentaren und Lizenzen. Anwendung von Vorschlägen für die Architektur des Frameworks                    \\
  \hline
  ChatGPT         & \url{https://chat.openai.com}             &
  Formulierungshilfe für englische Textpassagen, Korrekturvorschläge für Grammatik, Umformulierung komplexer Sätze. Rechtschreibunterstützung. Anfertigung von mathematischen Zeichungen und PlantUML Diagrammen auf Basis von Bildern. Erstellung von Formeln in \LaTeX auf Basis von Paperinformationen. \\
  \hline
  DeepL           & \url{https://deepl.com}                   &
  Überprüfung der englischen Grammatik und Rechtschreibung im gesamten Dokument.                                                                                                                                                                                                                           \\
  \hline
  Research Rabbit & \url{https://researchrabbitapp.com/}      &
  Literaturrecherche und Management.                                                                                                                                                                                                                                                                       \\
  \hline
\end{tabular}

\section*{Verwendungsumfang}

Die generativen KI-Werkzeuge wurden ausschließlich unterstützend eingesetzt:

\begin{itemize}
  \item \textbf{GitHub Copilot}: Gelegentliche Nutzung zur Unterstützung bei der Formatierung und Einrückung von Code sowie zur Inspiration bei der Lösung spezifischer Implementierungsfragen. Alle Vorschläge wurden kritisch geprüft und nur übernommen, wenn sie den eigenen Qualitätsansprüchen entsprachen.

  \item \textbf{ChatGPT}: Vereinzelte Nutzung zur sprachlichen Überarbeitung einzelner englischer Textpassagen sowie zur Klärung grammatikalischer Fragen. Die inhaltliche Ausarbeitung und Argumentation erfolgte eigenständig.

  \item \textbf{DeepL}: Diente punktuell zur Überprüfung der englischen Grammatik und Rechtschreibung in ausgewählten Abschnitten. Korrekturvorschläge wurden nur übernommen, sofern sie den beabsichtigten Inhalt nicht veränderten.

  \item \textbf{Research Rabbit}: Wurde unterstützend für die Literaturrecherche und -verwaltung eingesetzt.
\end{itemize}

\section*{Hinweis zur Eigenleistung}

Die in dieser Arbeit dargestellten Analysen, Konzepte und Ergebnisse wurden eigenständig erarbeitet. Die genannten Werkzeuge dienten ausschließlich der technischen und sprachlichen Unterstützung. Die kritische Auseinandersetzung mit dem Forschungsthema sowie die inhaltliche Ausgestaltung der Arbeit erfolgten unabhängig von den eingesetzten Hilfsmitteln.

\section*{Kennzeichnung generierter Inhalte}

Alle durch KI-Werkzeuge vorgeschlagenen Formulierungen, die substantiell in den Text übernommen wurden, sind durch das Symbol \ai{Text} gekennzeichnet. Geringfügige grammatikalische oder stilistische Anpassungen wurden nicht separat markiert, da sie den Inhalt nicht wesentlich verändert haben.

Inhalte aus wissenschaftlichen Quellen und eigene Gedanken stellen den überwiegenden Teil der vorliegenden Arbeit dar. Die Verwendung generativer Werkzeuge diente ausschließlich der sprachlichen und technischen Unterstützung und hat die eigene kritische Auseinandersetzung mit dem Forschungsthema nicht ersetzt.
