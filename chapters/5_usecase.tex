\chapter{Testing}
\label{chap:case-study}

\section{Application Scenario and Data Basis}
% Description of the production system (e.g., IoT Factory)
% Available event data and their characteristics
% Data pre-processing and analysis
% Preprocessing and Data Cleaning: Handling missing data, unifying part IDs, and filtering out irrelevant process steps (→ Reference to 4.2)

\section{Automatically Generated Digital Twin}
% Model generation process
% Model properties and parameters
% Aspects of automatic model generation in practice (→ reference to 3.2)

\section{Validation Experiments}
% Experimental design
% Execution of automated validation
% Error injection and model adjustment
% Feature Engineering: Creation of features such as "duration," "sequence_number," and periodic time features (e.g., "day_of_week_sin," "hour_of_day_cos") (→ Reference to 4.3)
% Machine Learning Models: Implementation of a Decision Tree Classifier and an xLSTM model for validation (→ Reference to 4.3)
% Empirical verification of theoretical V\&V concepts (→ reference to 2.2 and 4.3)

\section{Results and Interpretation}
% Model accuracy and reliability
% Detection rate of artificially injected errors
% Analysis of validation metrics (→ Reference to 4.4)
% Discrepancies and Anomalies: Identification of discrepancies between real and simulated data (→ Reference to 7.3)
% Evaluation of results in the context of V\&V theory (→ reference to 2.2)

\section{Comparison with Manual Validation Methods}
% Effort analysis
% Quality comparison
% Cost-benefit analysis
% Empirical evidence for the advantages of automated V\&V (→ reference to 3.2 and 7.3)