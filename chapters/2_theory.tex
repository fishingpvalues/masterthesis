\chapter{Theoretical Foundation}
\label{chap:theory}

\section{Material Flow Planning and Simulation}
\label{sec:material-flow}
\subsection{Basic Concepts}
% Content goes here
% - Basic concepts

\subsection{Processes and Resources}
% Content goes here
% - Processes and resources

\subsection{Production Planning and Control}
% Content goes here
% - Production planning and control

\subsection{Relevant KPIs and Metrics}
% Content goes here
% - Relevant KPIs and metrics (→ Reference to 4.4: Metrics for model evaluation)

\section{Digital Twin: Definition and Concepts}
\label{sec:digital-twin}
\subsection{Types of Digital Twins}
% Content goes here
% - Types of digital twins

\subsection{Data-Driven Digital Twins}
% Content goes here
% - Data-Driven Digital Twins (→ Reference to 3.2: Automatic model generation)

\subsection{Automatically Generated Digital Twins}
% Content goes here
% - Automatically generated digital twins
\subsection{Definitions and Differences from Classical Simulation Literature}
% Content goes here

\section{Process Mining and Event Logs}
\label{sec:process-mining}

\subsection{Core Concepts}
% Content goes here
% - Standard formats and their importance for automated validation
% Standards aus Will van der Aalst Buch beschreiben die hier einschlägig sind

\subsection{Object-Centric Event Logs as a Data Basis}
\label{sec:object-centric-event-logs}
% Content goes here
% - Object-centric event logs as a data basis (→ Reference to 4.2: Data-based validation strategy)


\subsection{Process Mining as Enabling Technology}
% Content goes here
% The importance of PM for model validation VVUQ
% - Process mining as a bridge between process data and model validation (→ Reference to 3.3)
% - PM helps in context of DT, VVUQ and Simulation



\section{VVUQ in the Context of Simulation-Based Digital Twins}
\label{sec:vvuq-sbdt}


\subsection{Historical Development of VVUQ Concepts}
% Content goes here
% - Historical development of V\&V concepts (→ Reference to 1.2)

\subsection{Requirements of VVUQ for Automatically Generated Models}
% Content goes here
% - Requirements specific to VVUQ in the context of automatically generated models (→ Reference to 3.1 and 7.2)

\subsection{Theoretical Argumentation for Merging Verification, Validation and Uncertainty Quantification}
% Content goes here
% - Theoretical argumentation for merging verification, validation and uncertainty quantification

\subsection{Machine Learning-Based Approaches}
% Content goes here
% - Classification methods for the detection of model deviations (→ Reference to 4.3)
% - Challenges in data preparation and feature selection
% - Discussion of previous ML approaches in the context of the V&V problem (→ Reference to 2.2 and 4.3)

\subsection{VVUQ in the Context of Digital Twins}
% Manual vs. automated approaches
% - Critical discussion of existing V&V definitions and methods (→ Reference back to 2.2)
% - Challenges in the validation of automatically generated models

\subsection{VVUQ in Corporate Practice}
% Content goes here
% - V\&V as a continuous process (→ Reference to 4.5: Online validation)



