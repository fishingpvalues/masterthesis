%\clearpage
%\markboth{Appendix}{Appendix}
\section{Appendix}


\label{Appendix_InputxGradient}
\subsection{Input x Gradient Attribution Plots}

\begin{figure}[H]
    \centering
    \includegraphics[width=1\linewidth]{pictures/captum/target0_attributions.png}
    \caption{Attribution plot for Product 1}
    \label{fig:appenidix_agg1_Captum}
\end{figure}

\begin{figure}[H]
    \centering
    \includegraphics[width=1\linewidth]{pictures/captum/target1_attributions.png}
    \caption{Attribution plot for Product 2}
    \label{fig:appenidix_agg2_Captum}
\end{figure}

\begin{figure}[H]
    \centering
    \includegraphics[width=1\linewidth]{pictures/captum/target2_attributions.png}
    \caption{Attribution plot for Product 3}
    \label{fig:appenidix_agg3_Captum}
\end{figure}

\begin{figure}[H]
    \centering
    \includegraphics[width=1\linewidth]{pictures/captum/target3_attributions.png}
    \caption{Attribution plot for Product 4}
    \label{fig:appenidix_agg4_Captum}
\end{figure}

\begin{figure}[H]
    \centering
    \includegraphics[width=1\linewidth]{pictures/captum/target4_attributions.png}
    \caption{Attribution plot for Product 5}
    \label{fig:appenidix_agg5_Captum}
\end{figure}

\begin{figure}[H]
    \centering
    \includegraphics[width=1\linewidth]{pictures/captum/target5_attributions.png}
    \caption{Attribution plot for Product 6}
    \label{fig:appenidix_agg6_Captum}
\end{figure}

\begin{figure}[H]
    \centering
    \includegraphics[width=1\linewidth]{pictures/captum/target6_attributions.png}
    \caption{Attribution plot for Product 7}
    \label{fig:appenidix_agg7_Captum}
\end{figure}

\begin{figure}[H]
    \centering
    \includegraphics[width=1\linewidth]{pictures/captum/target7_attributions.png}
    \caption{Attribution plot for Product 8}
    \label{fig:appenidix_agg8_Captum}
\end{figure}

\label{Appendix_SHAP}
\subsection{Product 1}


\begin{figure}[H]
    \centering
    \includegraphics[width=0.75\linewidth]{pictures//shap//SHAP_Bilder/SHAP_domainrand_action0.png}
    \caption{SHAP summary plot for action 0.
        Each point represents the local feature attribution value (Shapley value for feature and instance).
        Blue color indicates a low feature value, for binary variables this is 0, red indicates high feature values, for binary variables this is 1. A positive SHAP value is positively associated with the action, a negative SHAP value is negatively associated with the action. The features are displayed based on importance on average with decreasing importance from top to bottom.}
    \label{fig:SHAP_Action0}
\end{figure}


\begin{table}[ht!]
    \footnotesize
    \centering
    \caption{Top 10 variables for Input X Gradient and Product 1.}
    \label{tab:variables}
    \begin{tabularx}{\textwidth}{lXr}
        \toprule
        \textbf{Ranking} & \textbf{Variable}                        & \textbf{Value}        \\
        \midrule
        1                & buffer\_content\_duration\_prod1         & -0.09835365414619446  \\
        2                & last\_prod\_type\_is\_prod1              & 0.07755021005868912   \\
        3                & end\_of\_planning\_period\_demand\_prod5 & -0.06204185262322426  \\
        4                & end\_of\_planning\_period\_demand\_prod8 & -0.03671699017286301  \\
        5                & buffer\_content\_duration\_prod5         & 0.032430436462163925  \\
        6                & buffer\_content\_duration\_prod4         & 0.025369716808199883  \\
        7                & buffer\_fill\_level                      & 0.022456379607319832  \\
        8                & buffer\_content\_duration\_prod7         & 0.013850999064743519  \\
        9                & buffer\_content\_duration\_prod2         & 0.01368733774870634   \\
        10               & buffer\_content\_duration\_prod3         & -0.012776759453117847 \\
        \bottomrule
    \end{tabularx}
\end{table}
\FloatBarrier



\label{Appendix_Notproduced_Aggs}
\subsection{Product 6}

Product 6 was never produced.

Regarding SHAP, we can see that all variables have SHAP values around zero (\ref{fig:SHAP_Action5}).

\begin{figure}[H]
    \centering
    \includegraphics[width=0.75\linewidth]{pictures//shap//SHAP_Bilder/SHAP_domainrand_action5.png}
    \caption{SHAP summary plot for action 5.
        Each point represents the local feature attribution value (Shapley value for feature and instance).
        Blue color indicates a low feature value, for binary variables this is 0, red indicates high feature values, for binary variables this is 1. A positive SHAP value is positively associated with the action, a negative SHAP value is negatively associated with the action. The features are displayed based on importance on average with decreasing importance from top to bottom.}
    \label{fig:SHAP_Action5}
\end{figure}

These are the attributions for Input x Gradient. As again, the values are too small to interpret them:

\begin{table}[ht!]
    \footnotesize
    \centering
    \caption{Top 10 Variables for Input X Gradient and Product 6}
    \label{tab:top_variables_target5}
    \begin{tabularx}{\textwidth}{lXr}
        \toprule
        \textbf{Rank} & \textbf{Variable}                        & \textbf{Value} \\
        \midrule
        1             & end\_of\_planning\_period\_demand\_prod5 & -5.6928e-05    \\
        2             & buffer\_content\_duration\_prod6         & -3.2762e-05    \\
        3             & end\_of\_planning\_period\_demand\_prod8 & 3.2095e-05     \\
        4             & buffer\_content\_duration\_prod4         & -2.5057e-05    \\
        5             & buffer\_content\_duration\_prod7         & -1.7585e-05    \\
        6             & buffer\_content\_duration\_prod3         & 1.2410e-05     \\
        7             & buffer\_content\_duration\_prod1         & 1.2367e-05     \\
        8             & buffer\_content\_duration\_prod2         & 1.0998e-05     \\
        9             & buffer\_content\_duration\_prod8         & -1.0736e-05    \\
        10            & buffer\_content\_duration\_prod5         & -9.3647e-06    \\
        \bottomrule
    \end{tabularx}
\end{table}
\FloatBarrier

The attribution values are too low to be interpreted.

In summary, for both methods we can support the hypothesis that products that were not produced should play a less important role in xAI methods.

\subsection{Product 2}

Product 2 was never produced.

Regarding SHAP, SHAP values for most features are close to zero (\ref{fig:SHAP_Action1}). The trend for the most important variable, buffer content duration of prod8 is ambiguous: Higher buffer content of prod8 is plotted with negative, null, and positive SHAP values simultaneously, which cannot be interpreted.

However, there are few individual points indicating that a higher next 24h demand of prod8 is negatively associated with production of prod1, which might be one of the reasons why this prod was not produced. According to our hypothesis, this would mean that production of prod8 was critical; therefore, it was produced instead of prod2.

A fuller buffer content of prod7 is slightly associated with production of prod2. Even though, prod2 was not produced, this is still in line with our hypothesis. If prod7 is not critical (e.g., the buffer is full) that makes it more likely that another product, for example prod2 can be produced. However, the SHAP values are close to zero, which makes sense, because the product was not produced.

Last product type is prod5 is fourth important in this plot. We can see that if the last product produced was not prod5, it speaks against production of prod2. Prod5 was produced the most often; therefore, it makes sense that if prod2 had been produced, the chances would be higher for this to happen following prod5.

We can also see that buffer content of prod6 and production of prod1 is slightly negatively associated with production of prod2. The other SHAP values for the rest of features are close to zero.

In line with our hypothesis that the produced products should play are more important role in the xAI methods than the ones that were not produced, using visual inspection we can tell that asides from few outliers most features have SHAP values around zero and - besides buffer content duration for prod8 with an ambiguous pattern - no feature is strongly associated with production of prod2.
We can see that - again asides from the ambiguous pattern in buffer content duration for prod8 - the few features that show greater absolute SHAP values speak against production of prod2, which makes intuitive sense, because the product was not produced.

\begin{figure}[H]
    \centering
    \includegraphics[width=0.75\linewidth]{pictures//shap//SHAP_Bilder/SHAP_domainrand_action1.png}
    \caption{SHAP summary plot for action 1.
        Each point represents the local feature attribution value (Shapley value for feature and instance).
        Blue color indicates a low feature value, for binary variables this is 0, red indicates high feature values, for binary variables this is 1. A positive SHAP value is positively associated with the action, a negative SHAP value is negatively associated with the action. The features are displayed based on importance on average with decreasing importance from top to bottom.}
    \label{fig:SHAP_Action1}
\end{figure}


The following table shows the Input X Gradient attribution values:

\begin{table}[ht!]
    \footnotesize
    \centering
    \caption{Top 10 Variables for Input X Gradient and Product 2}
    \label{tab:top_variables_target1}
    \begin{tabularx}{\textwidth}{lXr}
        \toprule
        \textbf{Rank} & \textbf{Variable}                        & \textbf{Value} \\
        \midrule
        1             & last\_prod\_type\_is\_prod1              & -4.8995e-08    \\
        2             & buffer\_content\_duration\_prod2         & -4.6587e-08    \\
        3             & buffer\_content\_duration\_prod4         & -3.4221e-08    \\
        4             & buffer\_content\_duration\_prod1         & 2.4892e-08     \\
        5             & buffer\_content\_duration\_prod3         & -2.3662e-08    \\
        6             & buffer\_content\_duration\_prod6         & -2.0466e-08    \\
        7             & end\_of\_planning\_period\_demand\_prod5 & -1.4639e-08    \\
        8             & end\_of\_planning\_period\_demand\_prod7 & -1.3842e-08    \\
        9             & end\_of\_planning\_period\_demand\_prod8 & 1.1487e-08     \\
        10            & last\_prod\_type\_is\_prod5              & -9.3096e-09    \\
        \bottomrule
    \end{tabularx}
\end{table}
\FloatBarrier

The attribution values are too small to interpret.

In summary, for both methods the hypothesis that produced products should play a more important role can be supported.

\subsection{Product 3}

Product 3 was also never produced.

Regarding SHAP, we can see that asides from few individual points most variables have SHAP values around zero and do not strongly speak for production of prod3 (\ref{fig:SHAP_Action2}).

\begin{figure}[H]
    \centering
    \includegraphics[width=0.75\linewidth]{pictures//shap//SHAP_Bilder/SHAP_domainrand_action2.png}
    \caption{SHAP summary plot for action 2.
        Each point represents the local feature attribution value (Shapley value for feature and instance).
        Blue color indicates a low feature value, for binary variables this is 0, red indicates high feature values, for binary variables this is 1. A positive SHAP value is positively associated with the action, a negative SHAP value is negatively associated with the action. The features are displayed based on importance on average with decreasing importance from top to bottom.}
    \label{fig:SHAP_Action2}
\end{figure}

Regarding Input x Gradient, we derived the following attributions:

\begin{table}[ht!]
    \footnotesize
    \centering
    \caption{Top 10 Variables for Input X Gradient and Product 3}
    \label{tab:top_variables_target2}
    \begin{tabularx}{\textwidth}{lXr}
        \toprule
        \textbf{Rank} & \textbf{Variable}                        & \textbf{Value}        \\
        \midrule
        1             & last\_prod\_type\_is\_prod1              & -0.06599830090999603  \\
        2             & buffer\_content\_duration\_prod1         & 0.03362700715661049   \\
        3             & end\_of\_planning\_period\_demand\_prod5 & 0.02455740049481392   \\
        4             & buffer\_content\_duration\_prod6         & 0.02413715235888958   \\
        5             & last\_prod\_type\_is\_prod5              & -0.023769624531269073 \\
        6             & buffer\_content\_duration\_prod2         & -0.0206350926309824   \\
        7             & buffer\_content\_duration\_prod4         & 0.020487982779741287  \\
        8             & buffer\_content\_duration\_prod8         & -0.018504632636904716 \\
        9             & buffer\_content\_duration\_prod7         & -0.01190112717449665  \\
        10            & buffer\_fill\_level                      & 0.011671244166791439  \\
        \bottomrule
    \end{tabularx}
\end{table}
\FloatBarrier

If the product before was product 1, the chances that product 3 will be produced again are decreased.

In summary, for SHAP we can support that hypothesis that products that were not produced should play a less important role in xAI methods.

\subsection{Product 4}

Product 4 was never produced.

Regarding SHAP, we can see that asides from few individual points most variables have SHAP values around zero and do not strongly speak for production of prod4 (\ref{fig:SHAP_Action3}).

\begin{figure}[H]
    \centering
    \includegraphics[width=0.75\linewidth]{pictures//shap//SHAP_Bilder/SHAP_domainrand_action3.png}
    \caption{SHAP summary plot for action 3.
        Each point represents the local feature attribution value (Shapley value for feature and instance).
        Blue color indicates a low feature value, for binary variables this is 0, red indicates high feature values, for binary variables this is 1. A positive SHAP value is positively associated with the action, a negative SHAP value is negatively associated with the action. The features are displayed based on importance on average with decreasing importance from top to bottom.}
    \label{fig:SHAP_Action3}
\end{figure}

Regarding Input x Gradient, we derive

\begin{table}[ht!]
    \footnotesize
    \centering
    \caption{Top 10 Variables for Input X Gradient and Product 4}
    \label{tab:top_variables_target3}
    \begin{tabularx}{\textwidth}{lXr}
        \toprule
        \textbf{Rank} & \textbf{Variable}                        & \textbf{Value} \\
        \midrule
        1             & buffer\_content\_duration\_prod3         & -3.4675e-05    \\
        2             & buffer\_content\_duration\_prod4         & -2.5701e-05    \\
        3             & buffer\_content\_duration\_prod1         & 1.9520e-05     \\
        4             & last\_prod\_type\_is\_prod1              & -1.4845e-05    \\
        5             & buffer\_content\_duration\_prod6         & 1.0646e-05     \\
        6             & end\_of\_planning\_period\_demand\_prod8 & 8.7112e-06     \\
        7             & buffer\_content\_duration\_prod8         & -7.3491e-06    \\
        8             & buffer\_content\_duration\_prod5         & 7.1326e-06     \\
        9             & end\_of\_planning\_period\_demand\_prod7 & -4.4811e-06    \\
        10            & buffer\_content\_duration\_prod2         & -1.7606e-06    \\
        \bottomrule
    \end{tabularx}
\end{table}
\FloatBarrier

The attribution values are too small to interpret.

In summary, for both methods we can support the that hypothesis products that were not produced should play a less important role in xAI methods.

\subsection{Product 7}
\begin{figure}[H]
    \centering
    \includegraphics[width=0.75\linewidth]{pictures//shap//SHAP_Bilder/SHAP_domainrand_action6.png}
    \caption{SHAP summary plot for action 6.
        Each point represents the local feature attribution value (Shapley value for feature and instance).
        Blue color indicates a low feature value, for binary variables this is 0, red indicates high feature values, for binary variables this is 1. A positive SHAP value is positively associated with the action, a negative SHAP value is negatively associated with the action. The features are displayed based on importance on average with decreasing importance from top to bottom.}
    \label{fig:SHAP_Action6}
\end{figure}

Regarding Input x Gradient, these are the top ten attributions:

\begin{table}[ht!]
    \footnotesize
    \centering
    \caption{Top 10 Variables for Input X Gradient and Product 7}
    \label{tab:top_variables_target6}
    \begin{tabularx}{\textwidth}{lXr}
        \toprule
        \textbf{Rank} & \textbf{Variable}                        & \textbf{Value}        \\
        \midrule
        1             & buffer\_content\_duration\_prod8         & 0.06717950850725174   \\
        2             & buffer\_content\_duration\_prod7         & -0.0441112294793129   \\
        3             & buffer\_fill\_level                      & -0.033247679471969604 \\
        4             & buffer\_content\_duration\_prod3         & 0.032851435244083405  \\
        5             & buffer\_content\_duration\_prod4         & -0.02102786675095558  \\
        6             & buffer\_content\_duration\_prod6         & -0.01731930486857891  \\
        7             & buffer\_content\_duration\_prod2         & 0.012098127976059914  \\
        8             & end\_of\_planning\_period\_demand\_prod5 & -0.010609464719891548 \\
        9             & end\_of\_planning\_period\_demand\_prod7 & -0.008218199014663696 \\
        10            & next\_24h\_demand\_prod7                 & 0.008046381175518036  \\
        \bottomrule
    \end{tabularx}
\end{table}
\FloatBarrier


\subsection{Product 8}

\begin{figure}[H]
    \centering
    \includegraphics[width=0.75\linewidth]{pictures//shap//SHAP_Bilder/SHAP_domainrand_action7.png}
    \caption{SHAP summary plot for action 7.
        Each point represents the local feature attribution value (Shapley value for feature and instance).
        Blue color indicates a low feature value, for binary variables this is 0, red indicates high feature values, for binary variables this is 1. A positive SHAP value is positively associated with the action, a negative SHAP value is negatively associated with the action. The features are displayed based on importance on average with decreasing importance from top to bottom.}
    \label{fig:SHAP_Action7}
\end{figure}

Regarding Input x Gradient, these were the attributions:

\begin{table}[ht!]
    \footnotesize
    \centering
    \caption{Top 10 Variables for Input X Gradient and Product 8}
    \label{tab:top_variables_target7}
    \begin{tabularx}{\textwidth}{lXr}
        \toprule
        \textbf{Rank} & \textbf{Variable}                        & \textbf{Value}        \\
        \midrule
        1             & buffer\_content\_duration\_prod8         & -0.17791211605072021  \\
        2             & buffer\_content\_duration\_prod5         & 0.13158348202705383   \\
        3             & end\_of\_planning\_period\_demand\_prod8 & 0.0651685819029808    \\
        4             & last\_prod\_type\_is\_prod5              & -0.06460312008857727  \\
        5             & buffer\_content\_duration\_prod3         & -0.0536046028137207   \\
        6             & buffer\_content\_duration\_prod6         & 0.04870598390698433   \\
        7             & next\_24h\_demand\_prod8                 & 0.04792417585849762   \\
        8             & buffer\_content\_duration\_prod7         & 0.0367254838347435    \\
        9             & end\_of\_planning\_period\_demand\_prod5 & -0.027777016162872314 \\
        10            & next\_24h\_demand\_prod5                 & -0.011823729611933231 \\
        \bottomrule
    \end{tabularx}
\end{table}
\FloatBarrier


\subsection{Robustness check}
Using week 42 from real-world production, we generated new plots using Input x Gradient and SHAP for this new data. Then, we analyzed if our hypotheses still hold true, in order to ensure robustness of our approach.

In week 42, product 4 batches were produced 37 times and product 5 batches were produced 30 times. There were only two cases where our rebuild-network did not match the original one.
%Abbildungen entweder hier oder only repository!
\begin{figure}
    \centering
    \includegraphics[width=1\linewidth]{appendix//Robustness_check_week_42/action3_week42.png}
    \caption{SHAP plot for product 4.}
    \label{fig:agg4-week42}
\end{figure}

\begin{table}[ht!]
    \footnotesize
    \centering
    \caption{Top 10 Variables for Input X Gradient prod4}
    \label{tab:top_variables_prod4}
    \begin{tabularx}{\textwidth}{lXr}
        \toprule
        \textbf{Rank} & \textbf{Variable}                        & \textbf{Value}          \\
        \midrule
        1             & end\_of\_planning\_period\_demand\_prod8 & -4.196784253451824e-09  \\
        2             & buffer\_content\_duration\_prod1         & -2.7213378217538775e-09 \\
        3             & buffer\_fill\_level                      & 2.443531599283233e-09   \\
        4             & buffer\_content\_duration\_prod3         & -1.8710877291994166e-09 \\
        5             & end\_of\_planning\_period\_demand\_prod5 & -1.3332452919456728e-09 \\
        6             & buffer\_content\_duration\_prod4         & -9.293499303453245e-10  \\
        7             & buffer\_content\_duration\_prod5         & 8.536242268597505e-10   \\
        8             & buffer\_content\_duration\_prod7         & 7.789259237611645e-10   \\
        9             & end\_of\_planning\_period\_demand\_prod7 & -3.3234726082298494e-10 \\
        10            & next\_24h\_demand\_prod5                 & -2.8125202167217367e-10 \\
        \bottomrule
    \end{tabularx}
\end{table}
\FloatBarrier


Except one outlier, more buffer content of product 8 makes production of product 4 more likely (criticality-hypothesis).
Except one outlier, if product 4 was produced last, it is more likely to be produced again (setup-effort-hypothesis).
For buffer content duration of product 4 the trend is ambiguous, but a fuller buffer seems to speak for production. While this goes against our criticality hypothesis, it is inline with minimizing setup efforts; product 4 was produced 37 times, 36 of these without interruption. If no other product was critical (most variables for demand are close to zero), it makes sense that the agent stuck to production of product 4, even though the buffer content was already fuller, in order to minimize setup times.
\begin{figure}
    \centering
    \includegraphics[width=1\linewidth]{appendix//Robustness_check_week_42/action4_week42.png}
    \caption{SHAP plot for product 5}
    \label{fig:agg5-week42}
\end{figure}

\begin{table}[ht!]
    \footnotesize
    \centering
    \caption{Top 10 Variables for Input X Gradient prod5}
    \label{tab:top_variables_prod5}
    \begin{tabularx}{\textwidth}{lXr}
        \toprule
        \textbf{Rank} & \textbf{Variable}                        & \textbf{Value}        \\
        \midrule
        1             & end\_of\_planning\_period\_demand\_prod8 & -0.031818896532058716 \\
        2             & last\_prod\_type\_is\_prod4              & -0.026379607617855072 \\
        3             & buffer\_fill\_level                      & 0.026135722175240517  \\
        4             & buffer\_content\_duration\_prod8         & 0.02558363787829876   \\
        5             & buffer\_content\_duration\_prod7         & 0.020351236686110497  \\
        6             & buffer\_content\_duration\_prod5         & -0.0160413458943367   \\
        7             & buffer\_content\_duration\_prod2         & 0.015320430509746075  \\
        8             & buffer\_content\_duration\_prod1         & -0.015021555125713348 \\
        9             & next\_24h\_demand\_prod8                 & -0.012931049801409245 \\
        10            & end\_of\_planning\_period\_demand\_prod5 & 0.011032014153897762  \\
        \bottomrule
    \end{tabularx}
\end{table}
\FloatBarrier


If product 5 was produced last, it is more likely to be produced again (setup-effort-hypothesis).
Fuller buffer content of product 8 makes production of prod5 more likely, while high demand of product 8 speaks against production of product 5 (criticality-hypothesis).
Similarly, less buffer content and more 24h demand of product 5 speak for its production (criticality-hypothesis).
