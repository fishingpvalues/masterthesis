\section{Conclusion}
In this case study, we investigated the application of state-of-the-art xAI techniques for a DRL-based scheduling model. We built upon an existing agent in a real-world flow production setting, focusing on enhancing the interpretability of the agent's decisions for domain experts. Our investigation addressed both method-specific questions (RQ1.1, RQ1.2) as well as organizational aspects (RQ2.1, RQ2.2). On the method side, we utilized two prominent xAI frameworks, SHAP (DeepSHAP) and Captum (Input x Gradient), to analyze the reasoning behind the scheduling decisions. On the organizational side, our proposed xAI approach is based on the workflow of \citeA{tchuente2024methodological}, serving as a general procedural model for applying xAI methods in business use cases.

In summary, our findings highlight several critical issues in the current xAI literature, including a lack of falsifiability and consistent terminology, insufficient consideration of domain knowledge, inadequate attention to the target audience or real-world scenarios, and a tendency to offer simple input-output explanations rather than causal interpretations. Moreover, we observed that existing workflows often lack sufficient detail on how xAI methods and their results can be integrated with domain-specific aspects.

To address these challenges, we introduced a hypotheses-based workflow with feedback loops. This workflow allows for the inspection of explanations to ensure they are consistent with domain knowledge and the reward hypotheses of the agent. Our results show that both DeepSHAP and Input x Gradient are well-suited to explain the behavior of the agent, provided the methods are systematically embedded in the proposed workflow. However, DeepSHAP proved to be slightly more effective in our use case, as it was able to differentiate all the agent's actions more clearly. We hypothesize that this xAI workflow may also be applicable to other DRL-based scheduling models and should be tested and further developed in future studies. Further work is needed to adapt these tools for non-technical users, such as factory personnel. One way to achieve this may be intuitive interfaces and explanations formats.



\section*{Funding}
This work was partly supported by the Ministry of Economic Affairs,
Industry, Climate Action and Energy of the State of North Rhine-Westphalia,
Germany, under the project SUPPORT (005-2111-0026)

\section*{Author Contributions (CRediT)}

Daniel Fischer: \textit{Methodology, Formal analysis, Investigation, Software, Writing - original draft};
Hannah M. Hüsener: \textit{Methodology, Formal analysis, Investigation, Software, Writing - original draft};
Felix Grumbach: \textit{Conceptualization, Supervision, Writing – review \& editing};
Lukas Vollenkemper: \textit{Conceptualization, Supervision, Writing – review \& editing};
Arthur Müller: \textit{Data curation, Software, Writing – review \& editing};
Pascal Reusch: \textit{Funding acquisition, Resources};

\section*{Declarations}
\subsection*{Conflict of interest}
The authors have no conflicts of interest to declare that are relevant to the content of this article.

\subsection*{Open Access}
This article is licensed under a Creative Commons Attribution 4.0 International License, which permits use, sharing, adaptation, distribution and reproduction in any medium or format, as long as you give appropriate credit to the original author(s) and the source, provide a link to the Creative Commons licence, and indicate if changes were made. The images or other third party material in this article are included in the article’s Creative Commons licence, unless indicated otherwise in a credit line to the material. If material is not included in the article’s Creative Commons licence and your intended use is not permitted by statutory regulation or exceeds the permitted use, you will need to obtain permission directly from the copyright holder. To view a copy of this licence, visit http://creativecommons.org/licenses/by/4.0/.